\chapter{Theoretische Grundlagen}
\label{chap:theoretische_grundlagen}
\section{Greensche Funktionen}
Die Greensche Funktion ist für zwei Operatoren definiert als 
\begin{align}
    G_{A,B} \left (\tau, \tau' \right ) &= - \frac{1}{\hbar} \langle T_s \left ( A \left (\tau \right ) B \left ( \tau' \right ) \right ) \rangle \\
    & = - \frac{1}{\hbar} \left(  \langle A \left (\tau \right ) B \left ( \tau' \right ) \rangle \symup{\Theta} \left ( \tau - \tau' \right) + s 
    \langle B \left ( \tau' \right ) A \left (\tau \right ) \rangle \symup{\Theta} \left ( \tau' - \tau \right)  \right ) \; \text{,} \label{eqn:greensfunction}
\end{align}
wobei $T_s$ der Zeitordnungsoperator, $\langle \ldots \rangle$ ein Erwartungswert und $\symup{\Theta} \left ( \tau' - \tau \right)$ die Heaviside-Funktion ist.\cite{anders-fkt}\cite{uhrig-fktzwei}
Der Parameter $s$ sorgt mit $s=+1$ für bosonische bzw. $s=-1$ für fermionische Operatoren für das richtige Vorzeichen.
Die imaginäre Zeit $\tau$ ist über die reale Zeit $t$ als $\tau = it$ definiert.\cite{uhrig-fktzwei}
Die Operatoren $A$ und $B$ Operatoren im Heisenbergild besitzen somit die Darstellung 
\begin{equation}
    A \left ( \tau \right ) = \symup{e}^{H \tau} A_\text{S} \symup{e}^{-H \tau} \label{eqn:heisenbergpic}
\end{equation}
mit $H$ als Hamiltonoperator und $A_\text{S}$ als Operator im Schrödingerbild. 
In Gleichung \eqref{eqn:heisenbergpic} wurde das reduzierte Planksche Wirkungsquantum breits auf eins gesetzt, was in den folgenden Abschnitten beibehalten wird.
Da in dieser Arbeit nur fermionische Systeme betrachtet werden, wird $s$ ab jetzt ohne weitere Bemerkungen auf $-1$ gesetzt.
Unter der Annahme, dass die partielle Ableitung von $A$ verschwindet, ist die zeitliche Enwticklung eines Operators 
$A$ durch die Heisenbergsche Bewewgungsgleichung gemäß 
\begin{equation}
\frac{\symup{d}}{\symup{d}t} A \left (\tau \right ) = i  [H, A] \iff \frac{\symup{d}}{\symup{d}\tau} ( A \left (\tau \right ) = [H, A] \label{eqn:heisenbergeom}
\end{equation}
gegeben.
In dieser Arbeit ist die Bewegungsgleichung für die Greensche Funktion \eqref{eqn:greensfunction} von großer Bedeutung, welche mittels 
partieller Ableitung nach der Zeit gewonnen werden kann.
Somit folgt
\begin{align*}
    \begin{split}
    \frac{\partial}{\partial \tau} G \left (\tau, \tau' \right) = 
    &- \left \langle \frac{\partial}{\partial \tau}A \left (\tau \right ) B \left ( \tau' \right ) \right \rangle
    \symup{\Theta} \left ( \tau - \tau' \right) -  \langle A \left (\tau \right ) B \left ( \tau' \right ) \rangle \symup{\delta} \left ( \tau - \tau' \right)\\
    &+ \left \langle B \left ( \tau' \right ) \frac{\partial}{\partial \tau}A \left (\tau \right ) \right \rangle \symup{\Theta} \left ( \tau' - \tau \right)
    -  \langle B \left ( \tau' \right ) A \left (\tau \right ) \rangle \symup{\delta} \left ( \tau' - \tau \right)
    \end{split}
    \\[2ex]
    \begin{split}
    = &- \left \langle [H,A] B \left ( \tau' \right ) \right \rangle
    \symup{\Theta} \left ( \tau - \tau' \right) -  \langle A \left (\tau \right ) B \left ( \tau' \right ) \rangle \symup{\delta} \left ( \tau - \tau' \right)\\
    &+ \left \langle B \left ( \tau' \right ) [H,A] \right \rangle \symup{\Theta} \left ( \tau' - \tau \right)
    -  \langle B \left ( \tau' \right ) A \left (\tau \right ) \rangle \symup{\delta} \left ( \tau' - \tau \right)
    \end{split}
    \\[2ex]
    =\, & G_{[H,A],B}(\tau, \tau') - \langle [A,B] \langle \symup{\delta} \left ( \tau - \tau' \right) \; \text{.}
\end{align*} 








Eine mögliche Struktur der Arbeit sieht wie folgt aus:

\begin{enumerate}
    \item \textbf{Einleitung}\\
        In der \emph{kurzen} Einleitung wird die Motivation für die Arbeit
        dargestellt und ein Einblick in die kommenden Kapitel gegeben.
    \item \textbf{Theoretische Grundlagen}\\
        Alles was an theoretischen Grundlagen benötigt wird, sollte auch eher kurz gehalten werden.
        Statt Grundlagenwissen zu präsentieren, eher auf die entsprechenden Lehrbücher verweisen.
        Etwa: Tiefer gehende Informationen zur klassischen Mechanik entnehmen Sie bitte \cite{kuypers}.
    \item \textbf{Ergebnisse} \\
        Der eigentliche Teil der Arbeit, das was getan wurde.
    \item \textbf{Zusammenfassung und Ausblick} \\
        Zusammenfassung der Ergebnisse, Optimierungsmöglichkeiten, mögliche weitergehende Untersuchungen.
\end{enumerate}

Die Gliederung sollte auf der einen Seite nicht zu fein sein, auf der anderen Seite
sollten sich klar unterscheidende Abschnitte auch kenntlich gemacht werden.

In der hier verwendeten \KOMAScript-Klasse \texttt{scrbook} ist die oberste Gliederungsebene,
die in der Bachelorarbeit verwendet werden sollte, das \texttt{\textbackslash chapter}.

Ein Kapitel sollte erst dann in tiefere Gliederungsebenen unterteilt werden, wenn es auch wirklich etwas zu unterteilen gibt. Es sollte keine Kapitel mit nur einem Unterkapitel (\texttt{\textbackslash section}) geben.

In dieser Vorlage ist die Tiefe des Inhaltsverzeichnisses auf \texttt{chapter} und \texttt{section} beschränkt. Möchten Sie diese Beschränkung aufheben, entfernen Sie den Befehl
\begin{verbatim}
            \setcounter{tocdepth}{1}
\end{verbatim}
aus der Präambel oder ändern Sie den Zahlenwert entsprechend. Das Inhaltsverzeichnis sollte für eine Bachelorarbeit auf eine Seite passen.
