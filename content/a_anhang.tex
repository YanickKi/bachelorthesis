\chapter{Berechung der reziproken Gittervektoren}
\label{chap:appendixA}
Ausgehend von den Gittervektoren des Realraums 
\begin{equation*}
    \vec{a}_1 = \frac{a}{2}\begin{pmatrix} 3 \\[4pt] \sqrt{3}  \end{pmatrix}, \quad
    \vec{a}_2 = \frac{a}{2}\begin{pmatrix} 3 \\[4pt] -\sqrt{3} \end{pmatrix}       
\end{equation*}   
können die reziproken Gittervektoren mittels $\underline{\underline{B}} = 2\pi \left ( \underline{\underline{A}}^\text{T} \right )^{-1} $.\cite{suter}
In der Matrix $underline{\underline{A}}$ sind die Gittervektoren in den Spalten aufgetragen.
Somit folgt 
\begin{equation*}
    \underline{\underline{B}} = 2 \pi a
    \begin{pmatrix}
        \frac{3}{2} & \frac{\sqrt{3}}{2}  \\[0.35em]
        \frac{3}{2} & \frac{-\sqrt{3}}{2}
    \end{pmatrix}^{\! -1} \; .
\end{equation*}
Zur Berechnung der Inversen kann die Cramersche Regel für $2\times 2$ Matritzen andewendet, womit die gesuchte Matrix zu
\begin{equation*}
    \underline{\underline{B}} = - \frac{4\pi}{a3\sqrt{3}}
    \begin{pmatrix}
        -\frac{\sqrt{3}}{2} & -\frac{\sqrt{3}}{2}  \\[0.35em]
        -\frac{3}{2} & \frac{3}{2}
    \end{pmatrix}
\end{equation*}
berechnet werden kann.\cite{Cramer}
Die Spalten sind die reziproken Gittervektoren mit 
\begin{equation*}
    \vec{b}_1 = \frac{2\pi}{a3\sqrt{3}}\begin{pmatrix} \sqrt{3} \\[4pt] 3  \end{pmatrix}, \quad
    \vec{b}_2 = \frac{2\pi}{a3\sqrt{3}}\begin{pmatrix} \sqrt{3}       \\[4pt] -3 \end{pmatrix}   \; .  
\end{equation*}   