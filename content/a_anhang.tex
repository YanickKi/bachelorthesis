\chapter{Anhang}
\section{Berechnung der reziproken Gittervektoren}
\label{sec:rez_lattivectors_calc}
Ausgehend von den Gittervektoren des Realraums 
\begin{equation*}
    \vec{a}_1 = \frac{a}{2}\begin{pmatrix} 3 \\[4pt] \sqrt{3}  \end{pmatrix}, \quad
    \vec{a}_2 = \frac{a}{2}\begin{pmatrix} 3 \\[4pt] -\sqrt{3} \end{pmatrix}       
\end{equation*}   
können die reziproken Gittervektoren mittels $\underline{\underline{B}} = 2\pi \left ( \underline{\underline{A}}^\text{T} \right )^{-1} $ bestimmt werden \cite{suter}.
In der Matrix $\underline{\underline{A}}$ sind die Gittervektoren in den Spalten aufgetragen.
Somit folgt 
\begin{equation*}
    \underline{\underline{B}} = 2 \pi a
    \begin{pmatrix}
        \frac{3}{2} & \frac{\sqrt{3}}{2}  \\[0.35em]
        \frac{3}{2} & -\frac{\sqrt{3}}{2}
    \end{pmatrix}^{\! -1} \; .
\end{equation*}
Zur Berechnung der Inversen kann die Cramersche Regel für $2\times 2$ Matritzen angewendet werden, womit sich die gesuchte Matrix zu
\begin{equation*}
    \underline{\underline{B}} = - \frac{4\pi}{a3\sqrt{3}}
    \begin{pmatrix}
        -\frac{\sqrt{3}}{2} & -\frac{\sqrt{3}}{2}  \\[0.35em]
        -\frac{3}{2} & \frac{3}{2}
    \end{pmatrix}
\end{equation*}
ergibt \cite{Cramer}.
Die Spalten sind die reziproken Gittervektoren mit 
\begin{equation*}
    \vec{b}_1 = \frac{2\pi}{3a}\begin{pmatrix} 1        \\[4pt]     \sqrt{3}  \end{pmatrix}, \quad
    \vec{b}_2 = \frac{2\pi}{3a}\begin{pmatrix} 1        \\[4pt] -   \sqrt{3} \end{pmatrix}   \; .  
\end{equation*}
\section{Berechnung der Dispersionsrelation von Graphen}
\label{sec:calc_dispersion}
Der Tight Binding Hamiltonian in einer nächsten-Nachbar Näherung für Graphen lautet
\begin{equation*}
   H= - t  \sum_{i=1}^N \sum_{j=1}^3
   \left ( c_{\text{A},\vec{l}_i}^\dagger c_{\text{B},\vec{l}_i+\vec{\delta}_j} + c_{\text{B},\vec{l}_i+\vec{\delta}_j}^\dagger c_{\text{A},\vec{l}_i} \right ) \; .
\end{equation*}
Dazu werden Fouriertransformationen 
\begin{equation*}
    c_{\text{A},\vec{l}_i} = \frac{1}{\sqrt{N}} \sum_{\vec{k}}^{1. \, \text{BZ}} \symup{e}^{\symup{i}\vec{k}\vec{l}_i} c_{\text{A}, \vec{k}} \; , 
    \quad c_{\text{B},\vec{l}_i+\vec{\delta}_j} = \frac{1}{\sqrt{N}} \sum_{\vec{k}}^{1. \, \text{BZ}} \symup{e}^{\symup{i}\vec{k}(\vec{l}_i+\vec{\delta}_j)} c_{\text{B}, \vec{k}}
\end{equation*}
angesetzt. 
Diese werden in den Hamiltonian eingesetzt, womit 
\begin{align*}
    H &= -\frac{t}{N} \sum_{ij} \sum_{\vec{k}\vec{k}'} ( \symup{e}^{-\symup{i}\vec{k}\vec{l}_i}c^\dagger_{A,\vec{k}} 
    \symup{e}^{\symup{i}\vec{k}'(\vec{l}_i+\vec{\delta}_j)}c_{B,\vec{k}'} + \symup{e}^{-\symup{i}\vec{k}'(\vec{l}_i+\vec{\delta}_j)} c^\dagger_{B,\vec{k}'} 
    \symup{e}^{\symup{i}\vec{k}\vec{l}_i}c_{A,\vec{k}}) \\
    &= -\frac{t}{N} \sum_{ij} \sum_{\vec{k}\vec{k}'} ( \symup{e}^{-\symup{i}(\vec{k}- \vec{k}')\vec{l}_i}\symup{e}^{\symup{i}\vec{k'}\vec{\delta}_j}c^\dagger_{A,\vec{k}} c_{B,\vec{k}'} + 
    \symup{e}^{\symup{i}(\vec{k}- \vec{k}')\vec{l}} \symup{e}^{-\symup{i}\vec{k'}\vec{\delta}_j} c^\dagger_{B,\vec{k}'}c_{A,\vec{k}}) \\
    &= -t \sum_{j\vec{k}} ( \symup{e}^{\symup{i}\vec{k}\vec{\delta}_j}c^\dagger_{A,\vec{k}} c_{B,\vec{k}} + 
    \symup{e}^{-\symup{i}\vec{k}\vec{\delta_j}} c^\dagger_{B,\vec{k}}c_{A,\vec{k}})
\end{align*}
folgt.
In dem letzten Schritt wurde $\sum_{i}^N \symup{e}^{-\symup{i}(\vec{k}- \vec{k}')\vec{l}_i} 
= \sum_{i}^N \symup{e}^{\symup{i}(\vec{k}- \vec{k}')\vec{l}_i} = N\delta_{\vec{k}, \vec{k}'}$ ausgenutzt, so dass die Summe über $\vec{k}'$ verschwindet.
Der Hamiltonian kann in eine Matrixmultiplikation der Form 
\begin{equation*}
    H = \sum_{\vec{k}} \begin{pmatrix}
        c_{\text{A},\vec{k}}^{\dagger} & c_{\text{B},\vec{k}}^{\dagger}
    \end{pmatrix}
    \begin{pmatrix}
        0 & -t\sum_{j} \symup{e}^{\symup{i}\vec{k}\vec{\delta}_j}     \\
        -t\sum_{j} \symup{e}^{-\symup{i}\vec{k}\vec{\delta}_j} & 0     
    \end{pmatrix}
    \begin{pmatrix}
        c_{\text{A},\vec{k}} \\
        c_{\text{B},\vec{k}}
    \end{pmatrix}
\end{equation*}
gebracht werden.
Die Eigenwerte der auftretenden $2 \times 2$-Matrix bilden die Dispersionsrelation.
Diese Matrix kann gemäß
\begin{equation*}
    \begin{vmatrix}
        -\varepsilon_{\vec{k}} & -t\sum_{j} \symup{e}^{\symup{i}\vec{k}\vec{\delta}_j}     \\
        -t\sum_{j} \symup{e}^{-\symup{i}\vec{k}\vec{\delta}_j} & -\varepsilon_{\vec{k}}     
    \end{vmatrix} \stackrel{!}{=} 0 
    \iff \varepsilon_{\vec{k}}^2 - t^2 \left | \sum_{j} \symup{e}^{\symup{i}\vec{k}\vec{\delta}_j} \right |^2 = 0 
    \iff \varepsilon_{\vec{k}} = \pm t \left | \sum_{j} \symup{e}^{\symup{i}\vec{k}\vec{\delta}_j} \right |
\end{equation*}
diagonalisiert werden.
Die Summe ausgeschrieben ergibt
\begin{equation*}
    \varepsilon_{\vec{k}} = \pm t \sqrt{3+2 \cos \left ( \sqrt{3}ak_y \right )+2\cos \left ( \frac{3}{2}ak_x+\frac{\sqrt{3}}{2}ak_y \right ) + 2\cos \left ( \frac{3}{2}ak_x-\frac{\sqrt{3}}{2}ak_y \right ) } \; \; .
    \label{eqn:dispersion_graphene_calc}
\end{equation*}
