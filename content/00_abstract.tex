\thispagestyle{plain}

\section*{Kurzfassung}
In dieser Arbeit wird eine einzelne Mangan-Verunreinigung in Graphen mittels eines Tight Binding Modells in einer nächsten-Nachbarn Näherung
ohne die Coulomb-Wechselwirkung zwischen den Elektronen betrachtet.
Ein besonderes Augenmerk liegt dabei auf den Kopplungen zwischen den fünf $3d$-Orbitalen des Mangans und den $p_z$-Orbitalen der drei umliegenden 
Kohlenstoffatome, welche mittels Slater-Koster-Integralen beschrieben werden.
Das Ziel dieser Arbeit ist die Berechnung der Hybridisierungsfunktion der $3d$-Orbitale des Mangans. 
Einerseits wird diese mittels Bewegungsleichungen für Greensche Funktionen ermittelt.
Andererseits wird die Symmetrie des Problems mit Hilfe von gruppentheoretischen Überlegungen ausgenutzt, um die Hybridisierungsfunktion zu bestimmen.
Dabei wird gezeigt, welche $3d$-Orbitale mischen und es erfolgt der Nachweis eines effektiven Drei-Bänder Modells.
\section*{Abstract}
\begin{foreignlanguage}{english}
In the present work a single mangan impurity in graphene will be discussed with a nearest neighbour Tight Binding model, where
the Coulomb interaction will be neglected.
Particular attention is paid to the coupling between the five $3d$-orbitals of the manganese and the $p_z$-orbitals of the three
surrounding carbon atoms, which will be described by Slater-Koster integrals.
The aim of this thesis is the calculation of the hybridisation function of the $3d$-orbitals of the manganese.
One method is to use equations of motion for Green's functions. 
The hybridisation function is also determined by involving group theory to use the symmetry of the problem.
In the process the mixing of the $3d$-orbitals and a three band model will be verified. 
\end{foreignlanguage}