\chapter{Theoretische Grundlagen}
\label{chap:theoretische_grundlagen}
\section{Greensche Funktionen}
Die Greensche Funktion ist für zwei Operatoren definiert als 
\begin{align}
    G_{A,B} \left (\tau, \tau' \right ) &= - \frac{1}{\hbar} \langle T_s \left ( A \left (\tau \right ) B \left ( \tau' \right ) \right ) \rangle \\
    & = - \frac{1}{\hbar} \left(  \langle A \left (\tau \right ) B \left ( \tau' \right ) \rangle \symup{\Theta} \left ( \tau - \tau' \right) + s 
    \langle B \left ( \tau' \right ) A \left (\tau \right ) \rangle \symup{\Theta} \left ( \tau' - \tau \right)  \right ) \; \text{,} \label{eqn:greensfunction}
\end{align}
wobei $T_s$ der Zeitordnungsoperator, $\langle \ldots \rangle$ ein Erwartungswert und $\symup{\Theta} \left ( \tau' - \tau \right)$ die Heaviside-Funktion ist.\cite{greensfunction}
Der Parameter $s$ sorgt mit $s=+1$ für bosonische bzw. $s=-1$ für fermionische Operatoren für das richtige Vorzeichen.
<<<<<<< HEAD:content/02_theoretical_basics.tex
Der Mittelwert ist bezüglich der großkanonischen Gesamtheit durch 
\begin{equation*}
    \langle \ldots \rangle = \frac{1}{Z} \text{Sp}(\symup{e}^{-\beta H} \ldots)
\end{equation*}
gegeben.\cite{greensfunction}
%Die imaginäre Zeit $\tau$ ist über die reale Zeit $t$ als $\tau = it$ definiert.
Ein Operator $A$ besitzt im Heisenbergild die Darstellung 
\begin{equation*}
    A \left ( \tau \right ) = \symup{e}^{H \tau} A_\text{S} \symup{e}^{-H \tau}
\end{equation*}
Die imaginäre Zeit $\tau$ ist über die reale Zeit $t$ als $\tau = it$ definiert und ist, um die Konvergenz der Greenschen Funktion sicherzustellen 
auf $|\tau| < \beta$ beschränkt. \cite{anders-fkt}
Die Operatoren $A$ und $B$ Operatoren im Heisenbergild besitzen somit die Darstellung 
\begin{equation}
    A \left ( \tau \right ) = \symup{e}^{H \tau} A_\text{S} \symup{e}^{-H \tau} \label{eqn:heisenbergpic}
\end{equation}
mit $H$ als Hamiltonoperator und $A_\text{S}$ als Operator im Schrödingerbild. 
In Gleichung \eqref{eqn:heisenbergpic} wurde das reduzierte Planksche Wirkungsquantum breits auf eins gesetzt, was in den folgenden Abschnitten beibehalten wird.
Da in dieser Arbeit nur fermionische Systeme betrachtet werden, wird $s$ ab jetzt ohne weitere Bemerkungen auf $-1$ gesetzt.
Unter der Annahme, dass die partielle Ableitung von $A$ verschwindet, ist die zeitliche Enwticklung eines Operators 
$A$ durch die Heisenbergsche Bewewgungsgleichung gemäß 
\begin{equation}
\frac{\symup{d}}{\symup{d}t} A \left (\tau \right ) = i  [H, A] \iff \frac{\symup{d}}{\symup{d}\tau} A \left (\tau \right ) = [H, A] \label{eqn:heisenbergeom}
\end{equation}
gegeben.
In dieser Arbeit ist die Bewegungsgleichung für die Greensche Funktion \eqref{eqn:greensfunction} von großer Bedeutung, welche mittels 
partieller Ableitung nach der Zeit gewonnen werden kann.\cite{anders-fkt}
Somit folgt
\begin{align*}
    \begin{split}
    \frac{\partial}{\partial \tau} G_{A,B} \left (\tau, \tau' \right) = 
    &- \left \langle \frac{\partial}{\partial \tau}A \left (\tau \right ) B \left ( \tau' \right ) \right \rangle
    \symup{\Theta} \left ( \tau - \tau' \right) -  \langle A \left (\tau \right ) B \left ( \tau' \right ) \rangle \symup{\delta} \left ( \tau - \tau' \right)\\
    &+ \left \langle B \left ( \tau' \right ) \frac{\partial}{\partial \tau}A \left (\tau \right ) \right \rangle \symup{\Theta} \left ( \tau' - \tau \right)
    -  \langle B \left ( \tau' \right ) A \left (\tau \right ) \rangle \symup{\delta} \left ( \tau' - \tau \right)
    \end{split}
    \\[2ex]
    \begin{split}
    = &- \left \langle [H,A] B \left ( \tau' \right ) \right \rangle
    \symup{\Theta} \left ( \tau - \tau' \right) -  \langle A \left (\tau \right ) B \left ( \tau' \right ) \rangle \symup{\delta} \left ( \tau - \tau' \right)\\
    &+ \left \langle B \left ( \tau' \right ) [H,A] \right \rangle \symup{\Theta} \left ( \tau' - \tau \right)
    -  \langle B \left ( \tau' \right ) A \left (\tau \right ) \rangle \symup{\delta} \left ( \tau' - \tau \right)
    \end{split}
    \\[2ex]
    =\, & G_{[H,A],B}(\tau, \tau') - \langle [A,B] \rangle \symup{\delta} \left ( \tau - \tau' \right) \; \text{.}
\end{align*} 
Dabei ist  $\symup{\delta} (\tau-\tau') = \symup{\delta} (\tau'-\tau) $ die Deltadistribution, welche durch Ableitung der Heaviside-funktion gewonnen werden kann.
In der dritten Zeile wurde die Heisenbergsche Bewegungsgleichung \eqref{eqn:heisenbergeom} ausgenutzt.
Um eine einfachere Form der Bewegungsgleichung für die Greensche Funktion zu erhalten, wird diese fourier-transformiert.
Somit ergibt sich die Bewewgungsgleichung zu 
\begin{equation}
    zG_{A,B}(z) = \langle \{A,B\} \rangle - G_{[H,A],B}(z) \; \text{,} \label{eqn:fouriereom}
\end{equation} 
wobei der $\{ A,B \} = AB+BA$ der Antikommutator ist.
Eine wichtige Eigenschaft der Greenschen Funktion ist die Linearität, aus welcher
\begin{equation*}
    G_{(\alpha A + \beta B), C} = \alpha G_{A,C} + \beta G_{B,C}
\end{equation*}
folgt.
\begin{equation}
    G_{A,B}(z) = \int_0^\beta \symup{d}\tau \; \symup{e}^{z\tau}\, G_{A,B}(\tau)
\end{equation}

