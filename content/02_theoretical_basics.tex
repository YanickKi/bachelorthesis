\chapter{Theoretische Grundlagen}
\label{chap:theoretische_grundlagen}
\section{Greensche Funktionen}
Die Greensche Funktion ist für zwei Operatoren mit der zetlichen Entwicklung 
\begin{equation}
    A \left ( \tau \right ) = \symup{e}^{H \tau} A_\text{S} \symup{e}^{-H \tau}  \label{eqn:heisenbergpic}
\end{equation}
definiert als 
\begin{align}
    G_{A,B} \left (\tau, \tau' \right ) &= - \frac{1}{\hbar} \langle T_s \left ( A \left (\tau \right ) B \left ( \tau' \right ) \right ) \rangle \\
    & = - \frac{1}{\hbar} \left(  \langle A \left (\tau \right ) B \left ( \tau' \right ) \rangle \symup{\Theta} \left ( \tau - \tau' \right) + s 
    \langle B \left ( \tau' \right ) A \left (\tau \right ) \rangle \symup{\Theta} \left ( \tau' - \tau \right)  \right ) \; \text{,} \label{eqn:greensfunction}
\end{align}
wobei $H$ der Hamiltonoperator, $A_\text{S}$ ein Operator im Schöringerbild, $T_s$ der Zeitordnungsoperator, 
$\langle \ldots \rangle$ ein Erwartungswert und $\symup{\Theta} \left ( \tau' - \tau \right)$ die Heaviside-Funktion ist.\cite{greensfunction}
Der Parameter $s$ sorgt mit $s=+1$ für bosonische bzw. $s=-1$ für fermionische Operatoren für das richtige Vorzeichen.
Der Erwartungswert ist bezüglich der großkanonischen Gesamtheit durch 
\begin{equation*}
    \langle \ldots \rangle = \frac{1}{Z} \text{Sp}(\symup{e}^{-\beta H} \ldots)
\end{equation*}
gegeben.\cite{greensfunction}
Die Zustandssumme $Z$ stellt dabei einen Normierungsfaktor dar.
%Die imaginäre Zeit $\tau$ ist über die reale Zeit $t$ als $\tau = it$ definiert.
In Gleichung \eqref{eqn:heisenbergpic} wurde das reduzierte Planksche Wirkungsquantum breits auf eins gesetzt, was in den folgenden Abschnitten beibehalten wird.
Da in dieser Arbeit nur fermionische Systeme betrachtet werden, wird $s$ ab jetzt ohne weitere Bemerkungen auf $-1$ gesetzt.
Unter der Annahme, dass die partielle Ableitung von $A$ verschwindet, ist die zeitliche Entwicklung eines Operators 
$A$ durch die Heisenbergsche Bewegungsgleichung gemäß 
\begin{equation}
\frac{\symup{d}}{\symup{d}t} A \left (\tau \right ) = \symup{i}  [H, A] \iff \frac{\symup{d}}{\symup{d}\tau} A \left (\tau \right ) = [H, A] \label{eqn:heisenbergeom}
\end{equation}
gegeben.
In dieser Arbeit ist die Bewegungsgleichung für die Greensche Funktion \eqref{eqn:greensfunction} von großer Bedeutung, welche mittels 
partieller Ableitung nach der Zeit gewonnen werden kann.\cite{greensfunction}
Somit folgt
\begin{align*}
    \begin{split}
    \frac{\partial}{\partial \tau} G_{A,B} \left (\tau, \tau' \right) = 
    &- \left \langle \frac{\partial}{\partial \tau}A \left (\tau \right ) B \left ( \tau' \right ) \right \rangle
    \symup{\Theta} \left ( \tau - \tau' \right) -  \langle A \left (\tau \right ) B \left ( \tau' \right ) \rangle \symup{\delta} \left ( \tau - \tau' \right)\\
    &+ \left \langle B \left ( \tau' \right ) \frac{\partial}{\partial \tau}A \left (\tau \right ) \right \rangle \symup{\Theta} \left ( \tau' - \tau \right)
    -  \langle B \left ( \tau' \right ) A \left (\tau \right ) \rangle \symup{\delta} \left ( \tau' - \tau \right)
    \end{split}
    \\[2ex]
    \begin{split}
    = &- \left \langle [H,A] B \left ( \tau' \right ) \right \rangle
    \symup{\Theta} \left ( \tau - \tau' \right) -  \langle A \left (\tau \right ) B \left ( \tau' \right ) \rangle \symup{\delta} \left ( \tau - \tau' \right)\\
    &+ \left \langle B \left ( \tau' \right ) [H,A] \right \rangle \symup{\Theta} \left ( \tau' - \tau \right)
    -  \langle B \left ( \tau' \right ) A \left (\tau \right ) \rangle \symup{\delta} \left ( \tau' - \tau \right)
    \end{split}
    \\[2ex]
    =\; & G_{[H,A],B}(\tau, \tau') - \langle [A,B] \rangle \symup{\delta} \left ( \tau - \tau' \right) \; \text{.}
\end{align*} 
Dabei ist  $\symup{\delta} (\tau-\tau') = \symup{\delta} (\tau'-\tau) $ die Deltadistribution, welche durch Ableitung der Heaviside-funktion gewonnen werden kann.
In der dritten Zeile wurde die Heisenbergsche Bewegungsgleichung \eqref{eqn:heisenbergeom} ausgenutzt.
Um eine einfachere Form der Bewegungsgleichung für die Greensche Funktion zu erhalten, wird diese fourier-transformiert.
Somit ergibt sich die Bewewgungsgleichung zu 
\begin{equation}
    zG_{A,B}(z) = \langle \{A,B\} \rangle - G_{[H,A],B}(z) \; \text{,} \label{eqn:fouriereom}
\end{equation} 
wobei der $\{ A,B \} = AB+BA$ der Antikommutator ist.\cite{greensfunction}
Eine wichtige Eigenschaft der Greenschen Funktion ist die Linearität, aus welcher mit $\alpha, \beta \in \mathbb{C}$
\begin{equation*}
    G_{(\alpha A + \beta B), C} = \alpha G_{A,C} + \beta G_{B,C}
\end{equation*}
folgt.

\section{Tight Binding Modell}
\label{sec:tightbinding}
In der Tight Binding Näherung wird von stark gebundenen, lokalisierten Elektonen ausgegangen.\cite{Czycholl}
Dazu wird der volle Hamiltonian eines Elektrons, dessen atomaren Wellenfunktionen $\Psi_{lm}(\vec{r}-\vec{l}_i - \vec{R}_{\alpha})$ mit 
$l$ und $m$ als Drehimpulsquantenzahlen ist, im Festkörper
\begin{equation}
    H = \frac{\vec{p}^2}{2m} + \sum_{i\alpha} v(\vec{r}-\vec{l}_i - \vec{R}_{\alpha}) = \frac{\vec{p}^2}{2m} + v_{\vec{R}}(\vec{r})\label{eqn:electron_hamiltonian}
\end{equation}
betrachtet.\cite{Czycholl}
Der Vektor $\vec{R}_{\alpha}$ ist die Position innerhalb der Basis $\alpha$ in der Einheitszelle mit dem Gittervektor $\vec{l}_i$.
Aus den atomaren Wellenfunktionen lassen sich Blochzustände mit $\vec{k}$ als Wellenvektor
\begin{equation}
    \Psi^{\alpha}_{lm}(\vec{k}, \vec{r}) = \frac{1}{\sqrt{N}} \sum_{i} \symup{e}^{\symup{i}\vec{k}\vec{l}_i} \Psi_{lm}(\vec{r}-\vec{l}_i - \vec{R}_{\alpha}) 
\end{equation}
konstuieren, welche die Gitterperiodizität besitzen.\cite{SC_literature}
Für die Tight Binding Modellierung sind die Matrixelemente des Hamiltonians \eqref{eqn:electron_hamiltonian}
relevant, welche durch
\begin{equation*}
    \bra{\Psi^{\alpha}_{lm}} H \ket{\Psi^{\alpha'}_{l'm'}} = \varepsilon_{lm,l'm'}^{\alpha \alpha'} \braket*{\Psi^{\alpha}_{lm}}{\Psi^{\alpha'}_{l'm'}}
    - \frac{1}{N} \sum_{i\alpha \neq i' \alpha'} \symup{e}^{\symup{i}\vec{k}(\vec{l}_{i'}-\vec{l}_{i})}  t^{i\alpha,i'\alpha'}_{lm,l'm'}
\end{equation*}
gegeben sind.\cite{SC_literature}\cite{Czycholl}
Die Orbitalenergien im Kristallfeld sind durch $\varepsilon_{lm,l'm'}^{\alpha \alpha'}$ gegeben.\cite{SC_literature}
Dabei gilt
\begin{equation*}
    t^{i\alpha,i'\alpha'}_{lm,l'm'} = - \int \symup{d}^3r \; \overline{\Psi}_{lm}(\vec{r}-\vec{l}_i - \vec{R}_{\alpha}) 
    \left ( v_{\vec{R}} (\vec{r}) - v(\vec{r} - \vec{l}_{i'} - \vec{R}_{\alpha'} )   \right ) \Psi_{l'm'}(\vec{r}-\vec{l}_{i'} - \vec{R}_{\alpha'})  \; ,
\end{equation*}
wobei im stark lokalisierten Fall die Dreizentren-Beiträge vernachlässigt werden können und  
nur noch das Zweizentren-Integral 
\begin{equation*} 
    t^{i\alpha,i'\alpha'}_{lm,l'm'} = - \int \symup{d}^3r \; \overline{\Psi}_{lm}(\vec{r}-\vec{l}_i - \vec{R}_{\alpha}) 
    v(\vec{r} - \vec{l}_{i} - \vec{R}_{\alpha} ) \Psi_{l'm'}(\vec{r}-\vec{l}_{i'} - \vec{R}_{\alpha'}) 
\end{equation*}
übrig bleibt.\cite{SC_literature}
Dieses Zweizentren-Integral ist dann das für das Tight Binding Modell typische Hüpfmatrixelement.
Damit kann der Tight Binding Hamiltonian in zweiter Quantisierung bei Vernachlässigung der 
magnetischen Quantenzahl $m$ durch
\begin{equation}
    H = - \sum_{ii'} \sum_{\alpha \alpha'}\sum_{ll'} t^{i\alpha,i'\alpha'}_{ll'}  c_{il}^\dagger c_{i'l'}  \label{eqn:tight-binding-hamiltonian}
\end{equation}
angegeben werden.\cite{anders-fkt}
Dabei vernichtet(erzeugt) der Operator $c_{il\alpha}$($c_{il\alpha}^{\dagger}$ ) ein Elektron im Orbital $l$ in der Einheitszelle $i$,
innerhalb der Basis am Platz $\alpha$.
Je nach Modellannahme läuft die Summe dann über z.B. nächste oder übernächste Nachbarn.
\section{Slater-Koster-Integrale}
Jegliche Informationen dieses Abschnitts sind aus \cite{SC_literature} entnommen worden.
Die Slater-Koster-Integrale sind für zwei Orbitale, welche im Abstand $\vec{d}$ zueinander liegen, definiert als
\begin{equation}
    E_{lm,l'm'} = \int \symup{d}^3r \; \overline{\Psi}_{lm} (\vec{r}-\vec{d})
    V(\vec{r} - \vec{d}) \Psi_{l'm'} (\vec{r}) \; ,
\end{equation}
worin schon erkenntlich wird, dass diese nur von dem Abstand abhängen.
Dieses kann wiederrum in einige unabhängige Integrale $V_{ll'\beta}$ zerlegt werden, welche sich durch die Orbitale, $l$ und $l'$, sowie den Bindungen ($\sigma$, $\pi$, $\delta$) unterscheiden.
Als Beispiel würde sich bei dem Überlapp zwischen zwei $s$-Orbitalen das Slater-Koster-integral zu $E_{s,s} = V_{ss\sigma}$ ergeben, da dieser Überlapp komplett rotationssymmetrisch um die 
Bindungsachse ist.
Bei einem komplexeren Überlapp würden noch weiter Bindungen mit Vorfaktoren, den Richungskosinus, vorkommen.
Die Richtungskosinus sind Projektionen des normierten Abstandsvektors auf die Koordinatenachsen, wodurch die verschiedenen Bindungen erfasst werden können.