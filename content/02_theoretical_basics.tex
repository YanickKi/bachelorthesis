\chapter{Theoretische Grundlagen}
\label{chap:theoretische_grundlagen}
\section{Greensche Funktionen}
Die Greensche Funktion ist für zwei Operatoren mit der zetlichen Entwicklung 
\begin{equation}
    A \left ( \tau \right ) = \symup{e}^{H \tau} A_\text{S} \symup{e}^{-H \tau}  \label{eqn:heisenbergpic}
\end{equation}
definiert als 
\begin{align}
    G_{A,B} \left (\tau, \tau' \right ) &= - \frac{1}{\hbar} \langle T_s \left ( A \left (\tau \right ) B \left ( \tau' \right ) \right ) \rangle \\
    & = - \frac{1}{\hbar} \left(  \langle A \left (\tau \right ) B \left ( \tau' \right ) \rangle \symup{\Theta} \left ( \tau - \tau' \right) + s 
    \langle B \left ( \tau' \right ) A \left (\tau \right ) \rangle \symup{\Theta} \left ( \tau' - \tau \right)  \right ) \; \text{,} \label{eqn:greensfunction}
\end{align}
wobei $H$ der Hamiltonoperator, $A_\text{S}$ ein Operator im Schöringerbild, $T_s$ der Zeitordnungsoperator, 
$\langle \ldots \rangle$ ein Erwartungswert und $\symup{\Theta} \left ( \tau' - \tau \right)$ die Heaviside-Funktion ist.\cite{greensfunction}
Der Parameter $s$ sorgt mit $s=+1$ für bosonische bzw. $s=-1$ für fermionische Operatoren für das richtige Vorzeichen.
Der Erwartungswert ist bezüglich der großkanonischen Gesamtheit durch 
\begin{equation*}
    \langle \ldots \rangle = \frac{1}{Z} \text{Sp}(\symup{e}^{-\beta H} \ldots)
\end{equation*}
gegeben.\cite{greensfunction}
Die Zustandssumme $Z$ stellt dabei einen Normierungsfaktor dar.
%Die imaginäre Zeit $\tau$ ist über die reale Zeit $t$ als $\tau = it$ definiert.
In Gleichung \eqref{eqn:heisenbergpic} wurde das reduzierte Planksche Wirkungsquantum breits auf eins gesetzt, was in den folgenden Abschnitten beibehalten wird.
Da in dieser Arbeit nur fermionische Systeme betrachtet werden, wird $s$ ab jetzt ohne weitere Bemerkungen auf $-1$ gesetzt.
Unter der Annahme, dass die partielle Ableitung von $A$ verschwindet, ist die zeitliche Enwticklung eines Operators 
$A$ durch die Heisenbergsche Bewegungsgleichung gemäß 
\begin{equation}
\frac{\symup{d}}{\symup{d}t} A \left (\tau \right ) = \symup{i}  [H, A] \iff \frac{\symup{d}}{\symup{d}\tau} A \left (\tau \right ) = [H, A] \label{eqn:heisenbergeom}
\end{equation}
gegeben.
In dieser Arbeit ist die Bewegungsgleichung für die Greensche Funktion \eqref{eqn:greensfunction} von großer Bedeutung, welche mittels 
partieller Ableitung nach der Zeit gewonnen werden kann.\cite{greensfunction}
Somit folgt
\begin{align*}
    \begin{split}
    \frac{\partial}{\partial \tau} G_{A,B} \left (\tau, \tau' \right) = 
    &- \left \langle \frac{\partial}{\partial \tau}A \left (\tau \right ) B \left ( \tau' \right ) \right \rangle
    \symup{\Theta} \left ( \tau - \tau' \right) -  \langle A \left (\tau \right ) B \left ( \tau' \right ) \rangle \symup{\delta} \left ( \tau - \tau' \right)\\
    &+ \left \langle B \left ( \tau' \right ) \frac{\partial}{\partial \tau}A \left (\tau \right ) \right \rangle \symup{\Theta} \left ( \tau' - \tau \right)
    -  \langle B \left ( \tau' \right ) A \left (\tau \right ) \rangle \symup{\delta} \left ( \tau' - \tau \right)
    \end{split}
    \\[2ex]
    \begin{split}
    = &- \left \langle [H,A] B \left ( \tau' \right ) \right \rangle
    \symup{\Theta} \left ( \tau - \tau' \right) -  \langle A \left (\tau \right ) B \left ( \tau' \right ) \rangle \symup{\delta} \left ( \tau - \tau' \right)\\
    &+ \left \langle B \left ( \tau' \right ) [H,A] \right \rangle \symup{\Theta} \left ( \tau' - \tau \right)
    -  \langle B \left ( \tau' \right ) A \left (\tau \right ) \rangle \symup{\delta} \left ( \tau' - \tau \right)
    \end{split}
    \\[2ex]
    =\; & G_{[H,A],B}(\tau, \tau') - \langle [A,B] \rangle \symup{\delta} \left ( \tau - \tau' \right) \; \text{.}
\end{align*} 
Dabei ist  $\symup{\delta} (\tau-\tau') = \symup{\delta} (\tau'-\tau) $ die Deltadistribution, welche durch Ableitung der Heaviside-funktion gewonnen werden kann.
In der dritten Zeile wurde die Heisenbergsche Bewegungsgleichung \eqref{eqn:heisenbergeom} ausgenutzt.
Um eine einfachere Form der Bewegungsgleichung für die Greensche Funktion zu erhalten, wird diese fourier-transformiert.
Somit ergibt sich die Bewewgungsgleichung zu 
\begin{equation}
    zG_{A,B}(z) = \langle \{A,B\} \rangle - G_{[H,A],B}(z) \; \text{,} \label{eqn:fouriereom}
\end{equation} 
wobei der $\{ A,B \} = AB+BA$ der Antikommutator ist.\cite{greensfunction}
Eine wichtige Eigenschaft der Greenschen Funktion ist die Linearität, aus welcher mit $\alpha, \beta \in \mathbb{C}$
\begin{equation*}
    G_{(\alpha A + \beta B), C} = \alpha G_{A,C} + \beta G_{B,C}
\end{equation*}
folgt.

\section{Tight Binding Modell}
\label{sec:tightbinding}
Im Allgemeinen wird das Hüpfen eines Elektrons (bei Vernachlässigung des Spins) zwischen zwei Gitterpunkten durch den Tight-Binding Hamiltonian 
\begin{equation}
    H = - \sum_{in,i'n'} t_{n,n'}^{i,i'}  c_{in}^\dagger c_{i'n'}  \label{eqn:tight-binding-hamiltonian}
\end{equation}
beschrieben.\cite{SC_literature}
Dabei vernichtet(erzeugt) der Operator $c_{in}^{(\dagger)}$ ein Elektron im Orbital $n$ am Gitterpunkt $i$.
Die Hüpfparameter zu den betrachteten Orbitalen $\Psi_{lm}$ mit den Quantenzahlen $l$ und $m$, Basisvektoren $\vec{R}_\alpha$ und Gittervektoren
$\vec{l}_i$ sind durch die Slater-Koster Integrale \cite{PhysRev.94.1498}
\begin{equation} 
    V_{lm,l'm'}^{i\alpha,i'\alpha'} = -t_{lm,l'm'}^{i\alpha,i'\alpha'} = \int \symup{d}^3r \, \overline{\Psi}_{lm} (\vec{r} - \vec{R}_\alpha - \vec{l}_i)
    V(\vec{r} - \vec{R}_\alpha - \vec{l}_i) \Psi_{l'm'} (\vec{r} - \vec{R}_ {\alpha'} - \vec{l}_{i'})
\end{equation}
gegeben.\cite{SC_literature}
Hierbei ist $V(\vec{r} - \vec{R}_\alpha - \vec{l}_i)$ das atomate Potential, welches ein Elektron in der Einheitszelle $i$ am Platz $\alpha$ verspürt.
Diese hängen jedoch nur von dem Abstand $\vec{\delta}$ ab, wodurch sich diese zu 
\begin{equation}
    E_{lm,l'm'} = \int \symup{d}^3r \, \overline{\Psi}_{lm} (\vec{r}-\vec{\delta})
    V(\vec{r} - \vec{\delta}) \Psi_{l'm'} (\vec{r})
\end{equation}
ergeben.
Diese können in die verschiedenen Bindungstypen (in dieser Problemstellung $\pi$ und $\sigma$-Bindung) zerlegt werden, so dass diese nur noch von dem Abstandsvektor 
$\vec{\delta}$ und den Parametern $V_{ll'\beta}$ mit $l$ als Orbitale und $\beta$ als Bindungstyp ab.\cite{SC_literature}
Die Parameter $V_{ll'\beta}$ können hinterher als Fitparameter oder experimentell-bestimmte Werte fungieren.