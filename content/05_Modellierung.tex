\chapter{Berechnung}
\label{chap:berechnung}
\section{Berechnung der Greenschen Funktionen}
\label{sec:hamiltonian}
Der komplette Hamiltonian des betrachteten Graphens mit Mangandefekt lautet in Ortsdarstellung 
\begin{align}
   H &=  H_0 + H_\text{Def} + H_\text{Kop}\\
    &=- t \sum_{i,j}^N \left ( c_i^\dagger c_j + c_j^\dagger c_i \right ) \sum_m^5 \epsilon_m d_m^\dagger d_m
    + \sum_m^5 \sum_j^3 \left ( V_{mj} d_m^\dagger c_j + V^*_{mj} c_j^\dagger d_m  \right) \; \text{.} \label{eqn:full_Hamiltonian}
\end{align}
Der Term $H_0$ beschreibt mit dem Hüpfparameter $t$ das Hüpfen zwischen nächsten Nachbarn des ungestörten Graphengitters.
$H_\text{Def}$ umfasst die Störstelle lokal. Der Vernichter(Erzeuger) $d_m^{(\dagger)}$ vernichtet(erzeugt) dabei eine 
Elektron in dem $m$-ten $d$-Orbital mit der Einteilchenenergie $\epsilon_m$.
In dem letzten Term des Hamiltonians $H_\text{Kop}$ wird die Kopplung mit den Slater-Koster-integralen
$V_{mj}$ zwischen den 5 $3d$-Orbitale des Manganatoms mit dem $j$-ten Nachbarkohlenstoffatoms betrachtet.
Die Einträge der gesuchten Matrix $\underline{\underline{G}}$ werden mit Hilfe der Bewegunsgleichung \eqref{eqn:fouriereom} 
\begin{equation*}
    G_{d_m, d_{m'}^\dagger}(z) = \langle \{ d_m, d_{m'}^\dagger \} \rangle - G_{[H,d_m], d_{m'}^\dagger} (z)
\end{equation*}
bestimmt.
Dazu wird der Kommutator $[H,d_m]$ aufgrund der Linearität dessen in drei einzelne Kommutatoren unterteilt, so dass 
\begin{equation*}
    [H,d_m] = [H_0,d_m] + [H_\text{Def},d_m] +[H_\text{Kop},d_m] 
\end{equation*}
folgt. 