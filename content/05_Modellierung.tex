\chapter{Berechnung}
\label{chap:berechnung}
Der Hamiltonian des betrachteten Graphens mit Mangandefekt lautet
\begin{align*}
   H &=  H_0 + H_\text{Def} + H_\text{Kop}\\
    &=- t \sum_{i=1}^N \sum_{j=1}^N\left ( c_i^\dagger c_j + c_j^\dagger c_i \right )  + \sum_{m=1}^5 \epsilon_m d_m^\dagger d_m
    + \sum_{m=1}^5 \sum_{j=1}^3 \left ( V_{mj} d_m^\dagger c_j + \overline{V}_{mj} c_j^\dagger d_m \right )  \; \text{.} \numberthis \label{eqn:full_Hamiltonian}
\end{align*}
Die Summation über $N$ Basen in dem Term $H_0$ läuft nur über nächste Nachbarn.
Der Vernichter(Erzeuger) $c_i^{(\dagger)}$ vernichtet(erzeugt) ein Elektron im $p_z$-Orbital am Gitterplatz $i$.
Damit wird mit dem Hüpfparameter $t$ das Hüpfen zwischen nächsten Nachbarn des ungestörten Graphengitters beschrieben.
Aufgrund der Periodizität und der ausschließlichen Betrachtung des Hüpfens zwischen $p_z$-Orbitale, ist der  
Hüpfparameter $t$ für alle Summanden gleich.   
$H_\text{Def}$ umfasst die Störstelle. 
Der Vernichter(Erzeuger) $d_m^{(\dagger)}$ vernichtet(erzeugt) dabei ein Elektron in dem $m$-ten $3d$-Orbital mit der 
Orbitalenergie\cite{anders-fkt} $\epsilon_m$.
In dem letzten Term des Hamiltonians $H_\text{Kop}$ wird die Kopplung der $3d$-Orbitale des Manganatoms mit 
den $p_z$-Orbitalen der drei umliegenden Kohlenstoffatomen (siehe Abbildung \ref{fig:mangan_impurity_inplane}) aufgefasst.
Die auftretenden Hüpfparameter sind nicht wie zuvor bei $H_0$ für alle Summanden gleich, da fünf verschiedene Orbitale
des Manganatoms und die verschiedenen räumlichen Anordnungen der drei umliegenden Kohlenstoffatome einen Einfluss haben.
Somit bezeichnet $V_{mj}$ das Slater-Koster-Integral für das Hüpfen zwischen dem $m$-ten $3d$-Orbital des Manganatoms mit dem $p_z$-Orbital des 
$j$-ten Nachbarkohlenstoffatoms.
Da die Slater-Koster-Integrale reell sind, folgt $\overline{V}_{mj} = V_{mj}$.
Sofern es eindeutig ist, wird im Folgenden auf explizite Angabe des Startwerts des Laufindex und der oberen Grenze bei Summationen verzichtet.
\section{Berechnung der Slater-Koster-Integrale}
\label{sec:slaterkostercalc}
Zur Berechnung der Slater-Koster-Integrale werden die Richtungskosinus benötigt, welche durch
\begin{equation*}
    l = \frac{\vec{\delta} \cdot \hat{x}}{\left | \vec{\delta} \right |} \; , \quad
    m = \frac{\vec{\delta} \cdot \hat{y}}{\left | \vec{\delta} \right |} \; , \quad
    n = \frac{\vec{\delta} \cdot \hat{z}}{\left | \vec{\delta} \right |}
\end{equation*}
gegeben sind.
Dabei sind $\hat{x}$, $\hat{y}$ und $\hat{z}$ die Einheitsvektoren in die jeweiligen Richtungen. 
Der Betrag des Abstandsvektors (siehe Abschnitt \ref{sec:structure}) für das $i$-te Kohlenstoffatom um das Manganatom ist durch 
\begin{equation}
    \left | \vec{\delta} \right | = a \sqrt{\cos^2(\varphi_i) + \sin^2(\varphi_i) + \cot^2(\theta)} = a \sqrt{1+\cot^2(\theta)} = \frac{a}{\sin(\theta)} \label{eqn:distance}
\end{equation}
gegeben.
In Gleichung \eqref{eqn:distance} wurde die trigonometrische Beziehung $\sin(\theta) = \sqrt{1+\cot^2(\theta)}$ \cite{trig} mit der Beschränkung
$\theta \in [ 0, \sfrac{\pi}{2} ] $ ausgenutzt. 
Diese Beschränkung ist jedoch gerechtfertigt, da das Manganatom nur unterhalb des Graphens und mittig von den drei umliegenden Kohlenstoffatomen betrachtet wird.
Die Skalarprodukte ausgewertet und der Abstand eingesetzt ergibt
\begin{equation*}
    l = \cos(\varphi_i) \sin(\theta) \; , \quad
    m = \sin(\varphi_i) \sin(\theta) \; , \quad
    n = \cot(\theta) \sin(\theta) = \cos(\theta) \; .
\end{equation*}
Diese Ausdrücke für die Richtungskosinus werden in die Slater-Koster-Integrale für die Kopplung zwischen den drei umliegenden 
$p_z$-Orbitale der Kohlenstoffatome und den fünf $3d$-Orbitalen des Manganatoms (siehe Anhang \ref{TEMPLATE}) eingesetzt.
Diese ergeben die Tabelle \ref{tab:slaterkosters}.
\begin{table}
    \centering
    \caption{Slater-Koster-Integrale für den Überlapp der $p_z$-Orbitale des $j$-ten umliegenden
    Kohlenstoffatom und der fünf $3d$-Orbitale des Manganatoms in Abhängigkeit von dem Winkel $\theta$.}
    \label{tab:slaterkosters}
    \begin{tabular}{l c c c}
    & \multicolumn{3}{c}{$j$-tes Kohlenstoffatom}\\
    \cmidrule(lr){2-4}
    & {$1$} & {$2$} & {$3$} \\
    \midrule
    {$E_{z,xy}$      }  & {$0$}                                               & {$-\frac{3}{4}bV_{pd\sigma} + \frac{\sqrt{3}}{2}bV_{pd\pi}$}          & {$ \frac{3}{4}bV_{pd\sigma}-\frac{\sqrt{3}}{2}bV_{pd\pi}$}         \vspace{0.5cm} \\ 
    {$E_{z,xz}$      }  & {$\sqrt{3}fV_{pd\sigma} + hV_{pd\pi}$}              & {$-\frac{\sqrt{3}}{2}fV_{pd\sigma} - \frac{1}{2} hV_{pd\pi}$}         & {$-\frac{\sqrt{3}}{2}fV_{pd\sigma} - \frac{1}{2} hV_{pd\pi}$}      \vspace{0.5cm} \\
    {$E_{z,zy}$      }  & {$0$}                                               & {$ \frac{3}{2}fV_{pd\sigma}+\frac{\sqrt{3}}{2} hV_{pd\pi}$}           & {$-\frac{3}{2}fV_{pd\sigma}-\frac{\sqrt{3}}{2} hV_{pd\pi}$}        \vspace{0.5cm} \\
    {$E_{z,3z^2-r^2}$}  & {$q V_{pd\sigma}+\sqrt{3}bV_{pd\pi}$}               & {$q V_{pd\sigma}+\sqrt{3}bV_{pd\pi}$}                                 & {$q V_{pd\sigma}+\sqrt{3}bV_{pd\pi}$} \vspace{0.5cm} \\
    {$E_{z,x^2-y^2}$ }  & {$\frac{\sqrt{3}}{2}bV_{pd\sigma}-bV_{pd\pi}$}      & {$-\frac{\sqrt{3}}{4}bV_{pd\sigma}+\frac{1}{2}bV_{pd\pi}$}           & {$-\frac{\sqrt{3}}{4}bV_{pd\sigma}+\frac{1}{2}bV_{pd\pi}$}                       \\ 
    \bottomrule
    \end{tabular}
  \end{table}
Um die Übersichtlichkeit zu gewährleisten, wurden die von dem Winkel $\theta$ abhängigen Koeffizienten zu
\begin{align*}
b & \coloneq \sin^2(\theta) \cos(\theta)        & f &  \coloneq \cos^2(\theta) \sin(\theta)                             \\                     
h & \coloneq \sin(\theta)(1-2\cos^2(\theta))    & q &  \coloneq \cos^3(\theta) - \frac{1}{2}\sin^2(\theta) \cos(\theta)
\end{align*}
umdefiniert.
\section{Berechnung der Greenschen Funktionen}
\label{sec:calc_greensfunction}
Aufgrund der Gitterperiodizität kann für die Operatoren $c_i$ und $c_j$ eine Fouriertransformation der Form 
\begin{equation}
    c_i = \frac{1}{\sqrt{N}} \sum_{\vec{k}}^{1. \, \text{BZ}} \symup{e}^{i\vec{k}\vec{l}} c_{\text{A}, \vec{k}} \; , 
    \quad c_j = \frac{1}{\sqrt{N}} \sum_{\vec{k}}^{1. \, \text{BZ}} \symup{e}^{i\vec{k}(\vec{l}+\vec{\delta}_j)} c_{\text{B}, \vec{k}} \label{eqn:fourier_ladder}
\end{equation}
angesetzt werden.
Dabei sind $N$ die Anzahl der Basen, $\vec{l}$ der Gittervektor der Basis, $\vec{k}$ der Wellenvektor, $\vec{\delta}_j$ der Abstandsvektor
des $j$-ten Kohlenstoffatoms auf dem Untergitter B, welches der nächste Nachbar des Kohlenstoffatoms auf dem Untergitter A mit dem Gittervektor $\vec{l}$ ist 
und $c_{A, \vec{k}}$ bzw. $c_{B, \vec{k}}$ die Fourier-tranfsfromierten $p_z$-Orbitale\cite{anders-fkt} auf dem Untergitter A bzw. B.
Die $\vec{k}$-Werte werden aus der ersten Brillouin Zone genommen.\cite{anders-fkt}
Damit lässt sich der Hamiltonian des ungestörten Graphens zu 
\begin{align*}
    H_0 &= -\frac{t}{N} \sum_{j\vec{l}} \sum_{\vec{k}\vec{k}'} ( \symup{e}^{-i\vec{k}\vec{l}}c^\dagger_{A,\vec{k}} 
    \symup{e}^{i\vec{k}'(\vec{l}+\vec{\delta}_j)}c_{B,\vec{k}'} + \symup{e}^{-i\vec{k}'(\vec{l}+\vec{\delta}_j)} c^\dagger_{B,\vec{k}'} 
    \symup{e}^{i\vec{k}\vec{l}}c_{A,\vec{k}}) \\
    &= -\frac{t}{N} \sum_{j\vec{l}} \sum_{\vec{k}\vec{k}'} ( \symup{e}^{-i(\vec{k}- \vec{k}')\vec{l}}\symup{e}^{i\vec{k}\vec{\delta}_j}c^\dagger_{A,\vec{k}} c_{B,\vec{k}'} + 
    \symup{e}^{i(\vec{k}- \vec{k}')\vec{l}} \symup{e}^{-i\vec{k}\vec{\delta}_j} c^\dagger_{B,\vec{k}'}c_{A,\vec{k}}) \\
    &= -t \sum_{j\vec{k}} ( \symup{e}^{i\vec{k}\vec{\delta}_j}c^\dagger_{A,\vec{k}} c_{B,\vec{k}} + 
    \symup{e}^{-i\vec{k}\vec{\delta_j}} c^\dagger_{B,\vec{k}}c_{A,\vec{k}})
\end{align*}
schreiben.
In dem letzten Schritt wurde $\sum_{\vec{l}}^N \symup{e}^{-i(\vec{k}- \vec{k}')\vec{l}} 
= \sum_{\vec{l}}^N \symup{e}^{i(\vec{k}- \vec{k}')\vec{l}} = N\delta_{\vec{k}, \vec{k}'}$ ausgenutzt, so dass die Summe über $\vec{k}'$ verschwindet.
In $H_\text{Kop}$ werden die transformierten Erzeuger ohne weitere Vereinfachungen eingesetzt,  so dass der Gesamthamiltonian zu 
\begin{equation*}
    \begin{split}
        H = &-t \sum_{j \vec{k}} ( \symup{e}^{i\vec{k}\vec{\delta}_j}c^\dagger_{A,\vec{k}} c_{B,\vec{k}} + 
            \symup{e}^{-i\vec{k}\vec{\delta}_j} c^\dagger_{B,\vec{k}}c_{A,\vec{k}}) + \sum_m \epsilon_m d_m^\dagger d_m \\
            &+ \frac{1}{\sqrt{N}}\sum_{mj\vec{k}} ( V_{mj}  \symup{e}^{i\vec{k}(\vec{l}+\vec{\delta}_j)} d_m^\dagger c_{\text{B},\vec{k}} 
            + V_{mj} \symup{e}^{-i\vec{k}(\vec{l}+\vec{\delta}_j)}c^\dagger_{\text{B},\vec{k}} d_m )
    \end{split}
\end{equation*}
bestimmt werden kann.
Auf welchem Gitterplatz das Mangatom sitzt, wurde hierbei variabel gelassen.
Jedoch macht es in dieser Modellierung keinen Unterschied, welche Position das Manganatom einnimmt, wie sich im Verlauf der Berechnung auch zeigt.
Nun können die Einträge der gesuchten Matrix $\underline{\underline{G}}$ mit Hilfe der Bewegungsleichung \eqref{eqn:fouriereom} 
\begin{equation}
    zG_{d_m, d_{m'}^\dagger}(z) = \langle \{ d_m, d_{m'}^\dagger \} \rangle - G_{[H,d_m], d_{m'}^\dagger} (z) \label{eqn:eomgreenansatz}
\end{equation}
bestimmt werden.
Da sich das Argument der Greensfunktion nicht ändert, wird im folgenden drauf verzichtet, das Argument mit anzugeben.
Dazu wird der Kommutator $[H,d_m]$ aufgrund der Linearität dessen in drei einzelne Kommutatoren unterteilt, so dass
\begin{equation*}
    [H,d_m] = [H_0,d_m] + [H_\text{Def},d_m] +[H_\text{Kop},d_m] 
\end{equation*}
berechnet werden muss.
Somit lautet der gesamte Kommutator
\begin{equation}
    \begin{split}
    [H, d_m] = &-t \sum_{j\vec{k}} \left ( \symup{e}^{\symup{i}\vec{k} \vec{\delta}_j}      \left [ c^\dagger_{\text{A},\vec{k}}  
        c_{\text{B},\vec{k}}, d_m \right ] + \symup{e}^{-\symup{i}\vec{k} \vec{\delta}_j}   \left [ c^\dagger_{\text{B},\vec{k}}  
        c_{\text{A},\vec{k}}, d_m \right ]  \right ) \\
        &+\sum_{m'} \varepsilon_{m'} \left [ d^\dagger_{m'} d_{m'}, d_{m} \right ] \\
        &+ \frac{1}{\sqrt{N}} \sum_{m'j\vec{k}} \left ( V_{m'j} \symup{e}^{\symup{i}\vec{k} (\vec{l} + \vec{\delta}_j)}   
        \left [d^\dagger_{m'}c_{\text{B}, \vec{k}}, d_m \right ]
        +  V_{m'j} \symup{e}^{-\symup{i}\vec{k} (\vec{l} + \vec{\delta}_j)}   
        \left [c^\dagger_{\text{B},\vec{k}} d_{m'},  d_m \right ]
        \right ) \; . \label{eqn:commutator_H_dm}
    \end{split}
\end{equation} 
Die Kommutatoren werden im folgenden seperat ausgrechnet, damit diese hinterher nur noch eingesetzt werden müssen.
Für den ersten Kommutator folgt
\begin{equation}
    \left [ c^\dagger_{\text{A},\vec{k}}  c_{\text{B},\vec{k}}, d_m \right ]  = c^\dagger_{\text{A},\vec{k}}  c_{\text{B},\vec{k}}  d_m 
        -  d_m c^\dagger_{\text{A},\vec{k}}  c_{\text{B},\vec{k}}
        = - c^\dagger_{\text{A},\vec{k}} d_m c_{\text{B},\vec{k}} + c^\dagger_{\text{A},\vec{k}} d_m c_{\text{B},\vec{k}} = 0 \label{eqn:kommutatornull} \; .
\end{equation}
Bei dem Kommutator $\left [ c^\dagger_{\text{B},\vec{k}} c_{\text{A},\vec{k}}, d_m \right ]$ ist die Vorgehensweise und das Ergebnis gleich,
so dass auf ein explizites Vorrechnen verzichtet wird.
Weiterhin wird 
\begin{align*}
    \left [ d^\dagger_{m'} d_{m'}, d_{m} \right ] &= d^\dagger_{m'} d_{m'} d_{m} - d_{m} d^\dagger_{m'} d_{m'}  =
    - d^\dagger_{m'} d_{m} d_{m'} - \delta_{mm'} d_{m'} +  d^\dagger_{m'} d_{m} d_{m'} \\
    &= - \delta_{mm'} d_{m'}
\end{align*} 
berechnet.
Die Kommutatoren mit den Operatoren in $H_\text{Kop}$ ergeben 
\begin{align*}
    \left [d^\dagger_{m'} c_{\text{B},\vec{k}}, d_m \right ] &= d^\dagger_{m'} c_{\text{B},\vec{k}} d_m -d_m d^\dagger_{m'}c_{\text{B}, \vec{k}} 
    = - d^\dagger_{m'} d_m c_{\text{B},\vec{k}} - \delta_{mm'} c_{\text{B},\vec{k}} + d^\dagger_{m'} d_m c_{\text{B},\vec{k}} \\
    &= - \delta_{mm'} c_{\text{B},\vec{k}}
\end{align*}
und 
\begin{equation}
    \left [c^\dagger_{\text{B},\vec{k}} d_{m'},  d_m \right ] = c^\dagger_{\text{B},\vec{k}} d_{m'} d_m - d_m c^\dagger_{\text{B},\vec{k}} d_{m'} 
    = - c^\dagger_{\text{B},\vec{k}} d_{m} d_{m'} + c^\dagger_{\text{B},\vec{k}} d_m  d_{m'} = 0 \; . \label{eqn:commutatornulldagger}
\end{equation}
Werden nun alle eben berechneten Kommutatoren in \eqref{eqn:commutator_H_dm} eingesetzt, folgt mittels Auswertung der Kronecker-Delta
\begin{align*}
        [H, d_m] =& -\sum_{m'} \varepsilon_{m'} \delta_{mm'} d_{m'} - 
        \frac{1}{\sqrt{N}} \sum_{m'j\vec{k}} V_{m'j} \symup{e}^{\symup{i}\vec{k} (\vec{l} + \vec{\delta}_j)}  \delta_{mm'} c_{\text{B}, \vec{k}} \\
        = &- \varepsilon_m d_m - \frac{1}{\sqrt{N}} \sum_{j\vec{k}} V_{mj} \symup{e}^{\symup{i}\vec{k} (\vec{l} + \vec{\delta}_j)} c_{\text{B}, \vec{k}} \; .
        \numberthis \label{eqn:commutator_H_dm_eingesetzt}
\end{align*}
Die Summationen über $m'$ sind in Gleichung \eqref{eqn:commutator_H_dm_eingesetzt} aufgrund der Kronecker-Delta entfallen.
Der somit berechnete Kommutator kann in die ursprüngliche Bewegungsleichung \eqref{eqn:eomgreenansatz} eingesetzt werden, woraus mit 
$\langle \{ d_m, d_{m'}^\dagger \} \rangle = \delta_{mm'}$ und \eqref{eqn:commutator_H_dm_eingesetzt}
\begin{align*}
    z&G_{d_m, d_{m'}^\dagger} = \delta_{mm'} + \varepsilon_m G_{d_m, d^\dagger_{m'}} + \frac{1}{\sqrt{N}}\sum_{j\vec{k}} V_{mj} 
    \symup{e}^{\symup{i}\vec{k} (\vec{l} + \vec{\delta}_j)} G_{c_{\text{B},\vec{k}}, d^\dagger_{m'}} \\
    \iff \left (z-\varepsilon_m \right )&G_{d_m, d_{m'}^\dagger} = \delta_{mm'} + \frac{1}{\sqrt{N}}\sum_{j\vec{k}} V_{mj} 
    \symup{e}^{\symup{i}\vec{k} (\vec{l} + \vec{\delta}_j)} G_{c_{\text{B},\vec{k}}, d^\dagger_{m'}} \numberthis \label{eqn:greensumgestellt}
\end{align*}
folgt, da der Antikommutator das Kronecker-Delta $\delta_{mm'}$ ergibt und der Erwartungswert von einer Konstanen die Konstante selbst ist.
In Gleichung \eqref{eqn:greensumgestellt} kommt die Greensunktion $G_{c_{\text{B},\vec{k}}, d^\dagger_{m'}}$ vor, so dass diese
ebenfalls bestimmt werden muss, wofür 
\begin{equation}
    \begin{split}
        \left [ H, c_{\text{B},\vec{k}} \right ] = &-t \sum_{j\vec{k}'} \left ( \symup{e}^{\symup{i}\vec{k}' \vec{\delta}_j}  \left [ c^\dagger_{\text{A},\vec{k}'}  
        c_{\text{B},\vec{k}'}, c_{\text{B},\vec{k}} \right ] + \symup{e}^{-\symup{i}\vec{k}' \vec{\delta}_j}   \left [ c^\dagger_{\text{B},\vec{k}'}  
        c_{\text{A},\vec{k}'}, c_{\text{B},\vec{k}} \right ]  \right ) \\
        &+\sum_{m} \varepsilon_{m} \left [ d^\dagger_{m} d_{m}, c_{\text{B},\vec{k}}\right ] \\
        &+ \frac{1}{\sqrt{N}} \sum_{mj\vec{k}'} \left ( V_{mj} \symup{e}^{\symup{i}\vec{k}' (\vec{l} + \vec{\delta}_j)}   
        \left [d^\dagger_{m}c_{\text{B}, \vec{k}'}, c_{\text{B},\vec{k}} \right ]
        +  V_{mj} \symup{e}^{-\symup{i}\vec{k}' (\vec{l} + \vec{\delta}_j)}   
        \left [c^\dagger_{\text{B},\vec{k}'} d_{m},  c_{\text{B},\vec{k}} \right ]
        \right ) \label{eqn:commutator_H_cB}
    \end{split}
\end{equation}
benötigt wird.
Mit der selben Vorgehensweise wie in \eqref{eqn:kommutatornull} und \eqref{eqn:commutatornulldagger} verschwinden die 
Kommutator die kein $c^\dagger_{\text{B},\vec{k}'}$ enthalten, da die Operatoren ohne Berücksichtigung der Kommutatorrelation 
$\left \{ c_i, c^\dagger_j \right \} = \delta_{ij}$ verschoben werden können.
Somit bleiben nur noch 
\begin{align*}
    \left [ c^\dagger_{\text{B},\vec{k}'} c_{\text{A},\vec{k}'}, c_{\text{B},\vec{k}} \right ]
        &= c^\dagger_{\text{B},\vec{k}'} c_{\text{A},\vec{k}'} c_{\text{B},\vec{k}} - c_{\text{B},\vec{k}} c^\dagger_{\text{B},\vec{k}'} c_{\text{A},\vec{k}'} \\
        &= - c^\dagger_{\text{B},\vec{k}'} c_{\text{B},\vec{k}} c_{\text{A},\vec{k}'} - \delta_{\vec{k}\vec{k}'} c_{\text{A},\vec{k}'}
        + c^\dagger_{\text{B},\vec{k}'} c_{\text{B},\vec{k}} c_{\text{A},\vec{k}'} \\
        &=  - \delta_{\vec{k}\vec{k}'} c_{\text{A},\vec{k}'}
\end{align*}
und 
\begin{align*}
    \left [ c^\dagger_{\text{B},\vec{k}'} d_m, c_{\text{B},\vec{k}} \right ] &=
        c^\dagger_{\text{B},\vec{k}'} d_m c_{\text{B},\vec{k}} - c_{\text{B},\vec{k}} c^\dagger_{\text{B},\vec{k}'} d_m \\
        &= - c^\dagger_{\text{B},\vec{k}'} c_{\text{B},\vec{k}} d_m - \delta_{\vec{k}\vec{k}'} d_m + c^\dagger_{\text{B},\vec{k}'} c_{\text{B},\vec{k}} d_m \\
        &= - \delta_{\vec{k}\vec{k}'} d_m
\end{align*}
übrig.
somit kann der gesamte Kommutator \eqref{eqn:commutator_H_cB} als 
\begin{align*}
    \left [ H, c_{\text{B},\vec{k}} \right ]
     &= t \sum_{j,\vec{k}'} \symup{e}^{-\symup{i}\vec{k}' \vec{\delta}_j} \delta_{\vec{k}\vec{k}'} c_{\text{A},\vec{k}'}
      - \frac{1}{\sqrt{N}}\sum_{mj\vec{k}'} V_{mj} \symup{e}^{-\symup{i}\vec{k}' (\vec{l} + \vec{\delta}_j)} \delta_{\vec{k}\vec{k}'} d_m  \\
     &=  t \sum_{j} \symup{e}^{-\symup{i}\vec{k} \vec{\delta}_j} c_{\text{A},\vec{k}}
     - \frac{1}{\sqrt{N}}\sum_{mj} V_{mj} \symup{e}^{-\symup{i}\vec{k} (\vec{l} + \vec{\delta}_j)} d_m
\end{align*}
geschrieben werden, so dass dieser in die Bewegungsleichung für $G_{c_{\text{B},\vec{k}}, d^\dagger_{m'}}$ eingesetzt werden kann, woraus
\begin{align}
    G_{c_{\text{B},\vec{k}}, d^\dagger_{m'}} =\frac{1}{z} \left (- t \sum_{j} \symup{e}^{-\symup{i}\vec{k} \vec{\delta}_j} G_{c_{\text{A},\vec{k}},d^\dagger_{m'}} + 
    \frac{1}{\sqrt{N}} \sum_{mj} V_{mj} \symup{e}^{-\symup{i}\vec{k} (\vec{l}+ \vec{\delta}_j)} G_{d_m, d_{m'}^\dagger} \right )  \label{eqn:greensfunction_cB}
\end{align}
folgt.
In der Gleichung \eqref{eqn:greensfunction_cB} wird noch die Geensfunktion $G_{c_{\text{A},\vec{k}},d^\dagger_{m'}}$ benötigt, weswegen der Kommutator
$\left [H, c_{\text{A},\vec{k}} \right ]$ berechnet werden muss.
Mit der selbigen Begründung wie bei der Berechung von \eqref{eqn:commutator_H_cB} wird nur der Kommutator
\begin{align*}
    \left [ c^\dagger_{\text{A},\vec{k}'} c_{\text{B},\vec{k}'}, c_{\text{A},\vec{k}} \right ]  &= c^\dagger_{\text{A},\vec{k}'}  c_{\text{B},\vec{k}'} c_{\text{A},\vec{k}}
    - c_{\text{A},\vec{k}} c^\dagger_{\text{A},\vec{k}'} c_{\text{B},\vec{k}'} \\
    &= - c^\dagger_{\text{A},\vec{k}'}  c_{\text{A},\vec{k}} c_{\text{B},\vec{k}'} - \delta_{\vec{k}\vec{k}'} c_{\text{B},\vec{k}'} 
    + c_{\text{A},\vec{k}} c^\dagger_{\text{A},\vec{k}'} = - \delta_{\vec{k}\vec{k}'} c_{\text{B},\vec{k}'} 
\end{align*}
berechnet, womit 
\begin{equation}
    G_{c_{\text{A},\vec{k}},d^\dagger_{m'}} = - \frac{t \sum_{j}\symup{e}^{\symup{i}\vec{k}\vec{\delta}_j} G_{c_{\text{B},\vec{k}}, d^\dagger_{m'}}} {z} \label{eqn:greensfunction_cA}
\end{equation}
folgt.
Da keine neuen Greensfunktionen mehr auftauchen, sind alle nötigen Kommutatoren berechnet worden. 
Nun können die Greensfunktion ineinander eingesetzt werden, um die Matrix $\underline{\underline{G}}$ zu bestimmen.
Somit wird Gleichung \eqref{eqn:greensfunction_cA} in Gleichung \eqref{eqn:greensfunction_cB} eingesetzt, so dass sich die Greensfunktion
\begin{equation}
    G_{c_{\text{B},\vec{k}}, d^\dagger_{m'}} = \frac{z\sum_{mj} V_{mj} \symup{e}^{-\symup{i}\vec{k} (\vec{l}+\vec{\delta}_j)} G_{d_m, d_{m'}^\dagger}}
    {\sqrt{N}\left ( z^2-t^2\sum_j \symup{e}^{\symup{i}\vec{k}\vec{\delta}_j} \sum_j \symup{e}^{\symup{-i}\vec{k}\vec{\delta}_j} \right )} \label{eqn:finalgreencB}
\end{equation}
ergibt.
Um mehr Übersichtlichkeit zu gewähren wird die Definition
\begin{equation*}
     \mu_{\vec{k}} \coloneq t^2 \sum_j \symup{e}^{\symup{i}\vec{k}\vec{\delta}_j} \sum_j \symup{e}^{-\symup{i}\vec{k}\vec{\delta}_j} 
\end{equation*}  
und die Änderung der Notaton für die Einträge der Matrix $\underline{\underline{G}}$
\begin{equation*}
    G_{d_m, d^\dagger_{m'}} \to G_{mm'}
\end{equation*}
durchgeführt.
Das Ergebnis \eqref{eqn:finalgreencB} wird schlussendlich in Gleichung \eqref{eqn:greensumgestellt} eingesetzt, welche daraufhin in die gewünschte Form gebracht werden kann.
somit folgt
\begin{align*}
    &\left ( z- \varepsilon_m \right )  G_{mm'} = \delta_{mm'}+\sum_{j\vec{k}} V_{mj} \symup{e}^{\symup{i}\vec{k} ( \vec{l}+ \delta_j)} 
    \left ( \frac{z\sum_{mj} V_{mj} \symup{e}^{-\symup{i}\vec{k}(\vec{l} + \vec{\delta}_j)}G_{mm'}} {N (z^2-\mu_{\vec{k}})} \right ) \\
    &\iff \left ( z- \varepsilon_m - \frac{z}{N}\sum_{\vec{k}} \frac{\sum_j V_{mj} \symup{e}^{-\symup{i}\vec{k}(\vec{l} + 
    \vec{\delta}_j)} \sum_j V_{mj} \symup{e}^{-\symup{i}\vec{k}(\vec{l} + \vec{\delta}_j)}} {z^2-\mu_{\vec{k}}} \right ) G_{mm'} \\
    & \quad \quad \quad \quad \quad - \frac{z}{N}\sum_{n \neq m, \vec{k}} \frac{\sum_j V_{nj} \symup{e}^{\symup{i}\vec{k}(\vec{l} 
    + \vec{\delta}_j)} \sum_j V_{nj} \symup{e}^{-\symup{i}\vec{k}(\vec{l}+\vec{\delta}_j)}} {z^2-\mu_{\vec{k}}} G_{nm'} = \delta_{mm'} \numberthis \label{eqn:matrixmultiplication_green} \; .
\end{align*}
Hier wird, wie vorher erwähnt wurde, ersichtlich, dass die Position des Manganatoms keinen Einfluss auf das Ergebnis hat, da sich der Gittervektor 
$\vec{l}$ in Gleichung \eqref{eqn:matrixmultiplication_green} aufgrund des verschiedenen Vorzeichens im Exponenten weghebt.
In Gleichung \eqref{eqn:matrixmultiplication_green} wird nun eine Matrixmultiplikation 
\begin{equation*}
    \sum_n A_{mn}G_{nm'} = \delta_{mm'} \iff \underline{\underline{A}} \; \underline{\underline{G}} = \symbb{1}
\end{equation*}
erkennbar, wobei die Vorfaktoren vor $G_{mm'}$ die Hauptdiagonalelemente und die Vorfaktoren vor $G_{nm'}$ die Nebendiagonalelemente der Matrix 
$\underline{\underline{A}}$ bilden.
$\symbb{1} \in \symbb{R}^{5  \times 5}$ bezeichnet die Einheitsmatrix.
Somit bildet die Matrix $\underline{\underline{A}}$ die Inverse zu der Matrix $\underline{\underline{G}}$.
Die Einträge der Inversen sind mit $\mu_{\vec{k}}$ eingesetzt durch
\begin{equation}
    \left (  \underline{\underline{G}}^{-1} \right )_{mn} = \left ( z- \varepsilon_m \right ) \delta_{mn} - \frac{z}{N}\sum_{\vec{k}}
    \frac{\sum_j V_{mj}\symup{e}^{\symup{i}\vec{k}\vec{\delta}_j}\sum_j V_{nj}\symup{e}^{-\symup{i}\vec{k}\vec{\delta}_j}}
    {z^2-t^2\sum_j \symup{e}^{\symup{i}\vec{k}\vec{\delta}_j} \sum_j \symup{e}^{-\symup{i}\vec{k}\vec{\delta}_j}}
\end{equation}
gegeben.
Die Inverse der Matrix  $\underline{\underline{G}}$ besitzt die Struktur 
\begin{equation}
    \underline{\underline{G}}^{-1} = \underline{\underline{Z}} - \underline{\underline{E}} - \underline{\underline{\symup{\Delta}}} \; .
\end{equation}
Die diagonale Matrix $\underline{\underline{Z}} = z\symbb{1}$ ist durch die Skalierung der Einheitsmatrix mit $z$ gegeben, während die Energiematrix 
$\underline{\underline{E}} = \diag{\varepsilon_1, \varepsilon_2, \varepsilon_3, \varepsilon_4, \varepsilon_5}$
die Orbitalenergien $\varepsilon_m$ auf der Hauptdiagonale stehen hat.
Die Matrix $\underline{\underline{\symup{\Delta}}}$ ist die gesuchte Hybridisierungsfunktion mit 
\begin{equation*}
    \left ( \underline{\underline{\symup{\Delta}}} \right )_{mn} =  \frac{z}{N}\sum_{\vec{k}}
    \frac{\sum_j V_{mj}\symup{e}^{\symup{i}\vec{k}\vec{\delta}_j}\sum_j V_{nj}\symup{e}^{-\symup{i}\vec{k}\vec{\delta}_j}}
    {z^2-t^2\sum_j \symup{e}^{\symup{i}\vec{k}\vec{\delta}_j} \sum_j \symup{e}^{-\symup{i}\vec{k}\vec{\delta}_j}} \; .
\end{equation*}
\section{Linearkombinationen}
Aufgrund der dreizähligen Drehsymmetrie der Kohlenstoffatome um den Mangandefekt, können die Vernichter $c_i$
transformiert werden.
Dazu lassen sich die neuen $\tilde{c}_i$ durch Linearkombinationen der alten $c_i$ aufstellen, so dass das lineare Gleichungssystem mit komplexen Vorfaktoren 
\begin{align*}
    \tilde{c}_0 &= \frac{1}{\sqrt{3}}\left (c_1 + c_2 + c_3 \right ) \\
    \tilde{c}_1 &= \frac{1}{\sqrt{3}}\left (c_1+ \symup{e}^{\symup{i}\frac{2\pi}{3}} c_2 + \symup{e}^{\symup{i}\frac{4\pi}{3}}c_3 \right ) \\
    \tilde{c}_2 &= \frac{1}{\sqrt{3}}\left (c_1 + \symup{e}^{\symup{i}\frac{4\pi}{3}} c_2 + \symup{e}^{\symup{i}\frac{2\pi}{3}}c_3 \right ) \numberthis \label{eqn:sole} 
\end{align*}
gelöst werden muss. 
Der Faktor $\frac{1}{\sqrt{3}}$ sichert dabei die Normierung und die Erfüllung der Antikommutatorrelationen $\{\tilde{c}_i,\tilde{c}^\dagger_j\} = \delta_{ij}$.
Die Lösung dieses Gleichungssystems lautet 
\begin{align*}
    c_1 &= \frac{1}{\sqrt{3}}\left (\tilde{c}_0 + \tilde{c}_1 + \tilde{c}_2 \right ) \\
    c_2 &= \frac{1}{\sqrt{3}}\left (\tilde{c}_0 - \left( s \frac{1}{2} + \frac{{\sqrt{3}}}{2} \right) \tilde{c}_1 - \left ( \frac{1}{2} - \frac{{\sqrt{3}}}{2} \right ) \tilde{c}_2 \right ) \\
    c_2 &= \frac{1}{\sqrt{3}}\left (\tilde{c}_0 - \left(  \frac{1}{2} - \frac{{\sqrt{3}}}{2} \right) \tilde{c}_1 - \left ( \frac{1}{2} + \frac{{\sqrt{3}}}{2} \right ) \tilde{c}_2 \right ) 
    \; \text{.}     \numberthis \label{eqn:solutionsole}
\end{align*}
Diese Linearkombinationen \eqref{eqn:solutionsole} können gemeinsam mit den Slater-Koster-Integralen in den Hamiltonian $H_\text{Kop}$, der die Kopplung 
zwischen den $p_z$- und $3d$-Orbitalen beschreibt, eingesetzt werden.
Somit ergibt sich 
\begin{equation}
    \begin{split}
    H_\text{Kop} = \frac{1}{3} 
        &\left  (   \left   (-\frac{3}{4}\sqrt{3}   \symup{i}    b   V_{pd\sigma} + \frac{3}{2}  \symup{i}   b   V_{pd\pi} \right ) \tilde{c}^\dagger_1  
                    +\left  ( \frac{3}{4}\sqrt{3}   \symup{i}    b   V_{pd\sigma} - \frac{3}{2}  \symup{i}   b   V_{pd\pi} \right ) \tilde{c}^\dagger_2 \right )            d_1\\
        +&\left (    \left  ( \frac{3}{2}\sqrt{3}                           f   V_{pd\sigma} + \frac{3}{2}              h   V_{pd\pi} \right ) \tilde{c}^\dagger_1          
                    +\left  ( \frac{3}{2}\sqrt{3}                           f   V_{pd\sigma} + \frac{3}{2}              h   V_{pd\pi} \right ) \tilde{c}^\dagger_2 \right ) d_2\\
        +&\left (    \left  ( \frac{3}{2}\sqrt{3}               \symup{i}   f   V_{pd\sigma} + \frac{3}{2}  \symup{i}   h   V_{pd\pi} \right ) \tilde{c}^\dagger_1  
                    +\left  (-\sqrt{3}\frac{3}{2}               \symup{i}   f   V_{pd\sigma} - \frac{3}{2}  \symup{i}   h   V_{pd\pi} \right ) \tilde{c}^\dagger_2 \right ) d_3\\
        +&\left (   3 q V_{pd\sigma} + 3 \sqrt{3}  b V_{pd\pi} \right )    \tilde{c}^\dagger_0                                                                              d_4\\
        +&\left (   \left   ( \frac{3}{4}\sqrt{3}                           b   V_{pd\sigma} - \frac{3}{2}              b   V_{pd\pi} \right ) \tilde{c}^\dagger_1  
                    +\left  ( \frac{3}{4}\sqrt{3}                           b   V_{pd\sigma} - \frac{3}{2}              b   V_{pd\pi} \right ) \tilde{c}^\dagger_2 \right ) d_5\\
                    & + \text{h.c.} \; .
                 \end{split}
      \label{eqn:lcinhamiltonian}
\end{equation} 
Nun werden in in Gleichung \eqref{eqn:lcinhamiltonian} neue Linearkombinationen der Vernichter $d_i$ ersichtlich.
Diese lauten 
\begin{align*}
    \tilde{d}_0 &= \frac{1}{\sqrt{2}} (d_5-\symup{i}d_1) & \tilde{d}_1 &= \frac{1}{\sqrt{2}} (d_5-\symup{i}d_1) \\
    \tilde{d}_2 &= \frac{1}{\sqrt{2}} (d_2+\symup{i}d_3) & \tilde{d}_3 &= \frac{1}{\sqrt{2}} (d_2-\symup{i}d_3) \; .   \numberthis \label{eqn:newdi}
\end{align*}
Der Vorfaktor $\frac{1}{\sqrt{2}}$ dient erneut zur Normierung und gewährt ebenfalls die Antikommutatorrelation $\{ \tilde{d}_i, \tilde{d}^\dagger_j \} = \delta_{ij}$.
Wie in den Gleichungen \eqref{eqn:newdi} zu erkennen ist, mischen zweimal jeweils zwei Orbitale.
Einerseits mischt das Orbital $d_{xy}$ (Vernichter $d_1$) mit dem $d_{x^2-y^2}$-Orbital (Vernichter $d_5$).
Anderseits mischt das $d_{xz}$- mit dem $d_{zy}$-Orbital.
Im folgenden wird untersucht, welche Vernichter $\tilde{d}_i$ mit welchen $\tilde{c}_i$ koppeln.
Dazu werden die neuen Linearkombinationen für die Operatoren der Manganorbitale in den Hamiltonian $H_\text{Kop}$ eingesetzt, so dass sich dieser zu
\begin{equation}
            \begin{split}
                H_\text{Kop} = \frac{\sqrt{2}}{\sqrt{3}} 
                & \left ( \frac{3}{4}\sqrt{3} b V_{pd\sigma} + \frac{3}{2}  b   V_{pd\pi} \right )      \left ( \tilde{d}^\dagger_0 \tilde{c}_1 + \tilde{d}^\dagger_1 \tilde{c}_2 \right )   \\
            +    & \left ( \frac{3}{2}\sqrt{3} f V_{pd\sigma} + \frac{3}{2}  h   V_{pd\pi} \right )     \left ( \tilde{d}^\dagger_2 \tilde{c}_1 + \tilde{d}^\dagger_3 \tilde{c}_2 \right )   \\
            +    & \left ( \frac{3}{\sqrt{2}} q V_{pd\sigma} + \frac{\sqrt{3}}{\sqrt{2}} 3 b V_{pd\pi} \right ) \tilde{d}^\dagger_4 \tilde{c}_0 + \text{h.c.} \; .
            \end{split}
            \label{eqn:Hkopnice}
\end{equation}
vereinfacht. 