\chapter{Zusammenfassung und Ausblick}
\label{chap:Zusammenfassung_und_Ausblick}
In dieser Arbeit konnte die Hybridisierungsfunktion der $3d$-Orbitale des Mn bestimmt werden.
Dabei wurde die Höhe des Mn variabel gelassen.
Einerseits wurde dazu der Formalismus der Bewegungsgleichungen gewählt.
Anderseits wurde die Symmetrie des vorliegenden Problems genutzt und eine Basistransformation der $p_z$-Orbitale durchgeführt. 
Durch Einsetzen der SK-Integrale in den Hamiltonian konnten zwei Mischungen von jeweils zwei $3d$-Orbitalen gezeigt werden, welche nur 
an bestimmte $p_z$-Orbitale in der neuen Basis koppeln.
Somit wurde ein effektives Drei-Bänder Modell für die Ankopplung der $3d$-Orbitale des Mn an die Bandstruktur des Graphens nachgewiesen.
Da zweimal zwei Linearkombinationen und einmal eine Linearkombination der Vernichter der $3d$-Orbitale an nur einen Vernichter der Linearkombinationen
der $p_z$-Orbitale in der neuen Basis koppeln, zerfällt die Hybridisierungsfunktion in irreduzible Blöcke, welche auf der Hauptdiagonalen stehen. \\
Abschließend wurde die Abstandsabhängigkeit der SK-Integrale untersucht, wo sich herausstellte, welche SK-Integrale 
eine verschwindenden Einfluss bei gewissen Höhen haben.
Von Interesse wäre es zu prüfen, ob es zwischen dem verschwindenden Vorfaktor des SK-Integrals $E_{z,3r^2-r^2}$ $q$ bei der Höhe 
des Mn von $z=\qty{1}{\angstrom}$ und der tatsächlich ermittelten Höhe im Experiment (beschrieben in dem 
Artikel \cite{doi:10.1021/acsnano.1c00139}) von $z=\qty{1}{\angstrom}$ einen Zusammenhang gibt.\\
Aufbauend auf diese Arbeit kann die hier vernachlässigte Coulomb-Wechselwirkung der Elektronen in Betracht gezogen werden.
Außerdem wurden nur die $p_z$-Orbitale der C beachtet.
Jedoch wird die Bindung der $\sigma$-Orbitale der drei um die Fehlstelle liegenden C durch Entfernen des C
aufgebrochen, welche mit dem Mn ebenfalls hybdridisieren.
Analog zu den $p_z$-Orbitalen können Linearkombinationen der $\sigma$-Orbitale aufgestellt werden, welche an die selben 
$3d$-Orbitale des Mn koppeln, wobei hierbei von gebundenen Zuständen ausgegangen werden kann \cite{PhysRevB.97.155419}.
Da die $\sigma$-Orbitale ein nicht-verschwindendes Hüpfmatrixelement haben, spalten die drei Zustände auf.
Dabei werden zwei von den drei Elektronen in den Zustand niedrigster Energie aufgefüllt, welche eine Bindung erzeugen. 
Das übrige Elektron koppelt an das Mn, wobei diese Kopplung mit SK-Integralen beschrieben werden kann, was 
zu einer antiferromagnetischen Kopplung führt.
Damit könnten die Linearkombinationen der umliegenden $p_z$ Orbitale drei Abschirmkanäle bereitstellen, während der verbleibende $\sigma$-Zustand
ebenfalls ein Singulett mit einem der Mn-Orbitale bilden wird, so dass bei einer verschwindenden Temperatur $T \to 0$ ein Spin von 
$\sfrac{1}{2}$ übrig bleibt.