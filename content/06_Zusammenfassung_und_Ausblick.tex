\chapter{Zusammenfassung und Ausblick}
\label{chap:Zusammenfassung_und_Ausblick}
In dieser Arbeit konnte die Hybridisierungsfunktion der $3d$-Orbitale des Mn bestimmt werden.
Dabei wurde die Höhe des Mn variabel gelassen.
Einerseits wurde dazu der Formalismus der Bewegungsgleichungen gewählt.
Anderseits wurde die Symmetrie des vorliegenden Problems genutzt und eine Basistransformation der $p_z$-Orbitale durchgeführt. 
Durch Einsetzen der SK-Integrale in den Hamiltonian konnten zwei Mischungmen von jeweils zwei $3d$-Orbitalen gezeigt werden, welche nur 
an bestimmte $p_z$-Orbitale in der neuen Basis koppeln.
Somit wurde ein effektives Drei-Bänder Modell nachgewiesen, womit sich die Struktur der gesuchten Hybridisierungsfunktion vereinfacht hat. 
Da zweimal zwei Linearkombinationen und einmal eine Linearkombination der Vernichter der $3d$-Orbitale an nur einen Vernichter der $p_z$-Orbitale in der neuen Basis 
koppeln, zerfällt die Hybridisierungsfunktion in Blöcke, welche auf der Hauptdiagonalen stehen. 
Abschließend wurden die Abstandsabhängigkeit der SK-Integrale untersucht, wo sich herausstellte, welche SK-Integrale 
eine verschwindenen Einfluss bei gewissen Höhen haben.\\
Aufbauend auf diese Arbeit kann die hier vernachlässigte Coulomb-Wechselwirkung der Elektronen in Betracht gezogen werden.
Außerdem wurden nur die $p_z$-Orbitale der Kohlenstoffatome in Betracht gezogen.
Jedoch wird die Bindung der $\sigma$-Orbitale der drei um die Fehlstelle liegenden Kohlenstoffatome durch Entfernen des Kohlenstoffatoms
aufgebrochen, welche mit dem Mangan ebenfalls hybdridisieren.
Analog zur den $p_z$-Orbitalen können Linearkombinationen der $\sigma$-Orbitale aufgestellt werden, welche an die selben 
$3d$-Orbitale des Mn koppeln, wobei hierbei von gebundenen Zuständen ausgegangen werden kann.
Da die $\sigma$-Orbitale ein nicht-verschwindendes Hüpfmatrixelement haben, spalten die die drei Zustände auf.
Dabei werden zwei von den drei Elektronen in den Zustand niedrigster Energie aufgefüllt und eine Bindung erzeugen. 
Das übrige Elektron koppelt an das Mn, wobei diese Kopplung mit SK-Integralen beschrieben werden kann, womit 
es zu einer antiferromagnetischen Kopplung führt.